sucio.tex

% \begin{frame}[fragile]{Estudio de la heterogeneidad celular}
%   \vskip0.5em
%   Necesidad de herramientas para el estudio de la heterogeneidad celular.
%   % \Fontvi
%   \vskip-1em
%   \begin{overlayarea}{\linewidth}{1\textheight}
%   \begin{columns}
    
%     \begin{column}{0.65\textwidth}
%       \vskip-0.5em
%       \begin{alertblock}{Tradicionalmente}
%         \vskip1mm
%         A nivel de proteína mediante técnicas inmunohistoquímicas, inmunofluorescencencia y citometría de flujo. 
%       \end{alertblock}
%       \begin{alertblock}{Tecnologías de alto rendimiento}
%         \vskip1mm
%         A nivel transcriptómico mediante NGS: \textit{RNA-seq}.\\
%         \vskip1mm
%         Dos variantes:
%       \end{alertblock}
%     \end{column}

%     \begin{column}<2,3,4>{0.3\textwidth}
%       \vskip7mm
%       \begin{highlightblock}
%         \centering
%         \Fontvi{Pequeña combinación de marcadores génicos.}
%       \end{highlightblock}
%       \vskip1mm
%       \begin{highlightblock}
%         \centering
%         \Fontvi{Estatus funcional completo.}
%       \end{highlightblock}
%     \end{column}
%   \end{columns}
  
%   \vskip-1.5em

%   % Parte inferior
%   \begin{columns}
%     \begin{column}<2,3,4>{0.7\textwidth}
%       \Fontvi
%       \begin{itemize}
%         \item \textit{Bulk RNA-seq} (nivel tisular): los niveles de expresión corresponden al sumatorio de tipos celulares presentes en las muestras.
%         \vskip5mm
%         \item \textit{Single-cell RNA-seq} (nivel celular): los niveles de expresión corresponden a cada célula individual que compone la muestra.
%       \end{itemize}
%     \end{column}

%     \begin{column}{0.3\textwidth}
%       \begin{figure}
%         \centering
%         \hskip-4mm
%         \includegraphics<3,4>[width=4cm]{images/tumor_bw_arrow.png}\\
%         \hskip-4mm
%         \includegraphics<4>[width=4cm]{images/tumor_color_arrow.png}  
%       \end{figure}
      
%     \end{column}
%   \end{columns}
% \end{overlayarea}
% \end{frame}